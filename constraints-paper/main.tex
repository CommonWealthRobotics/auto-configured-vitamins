\documentclass{article}
\usepackage[utf8]{inputenc}
\usepackage{hyperref}
\usepackage{amsmath}
\usepackage{bera}
\usepackage{listings}
\usepackage{xcolor}
\usepackage{graphicx}
\usepackage{subcaption}
\usepackage{placeins}
\usepackage[paper=a4paper,margin=1.25in]{geometry}
\usepackage{bm}

\title{Arm Constraints}
\author{Kevin Harrington, Ryan Benasutti}
\date{\today}

\newcommand{\code}{\texttt}

\begin{document}

\maketitle

\FloatBarrier
\section{Arm Constraints}

\FloatBarrier
\subsection{Features}

The feature matrix of a motor module is given by
\begin{equation}
    F_m =
    \begin{bmatrix}
        \tau_{stall}^{(1)} & \tau_{stall}^{(2)} & \tau_{stall}^{(3)} \\[6pt]
        \omega_{free}^{(1)} & \omega_{free}^{(2)} & \omega_{free}^{(3)} \\[6pt]
        P^{(1)} & P^{(2)} & P^{(3)} \\[6pt]
        M^{(1)} & M^{(2)} & M^{(3)}
    \end{bmatrix}
\end{equation}

where $\tau_{stall}^{(i)}$ is the stall torque in Newton-meters for motor $i$,
$\omega_{free}^{(i)}$ is the free speed in radians per second for motor $i$,
$P^{(i)}$ is the price of motor $i$, and $M^{(i)}$ is the mass in kilograms of
motor $i$.

\FloatBarrier
\subsection{Required Tip Force and Velocity}

$V$ is the tip velocity (given as \texttt{requiredTipVelocityMeterPerSec}). $F$
is the tip force (given as \texttt{requiredTipForceNewtons}). $R_j$ is the r
parameter of link $j$ (the \texttt{dh-A} parameter). The arm is mounted $90
\deg$ off vertical.

The torque-speed curve of a motor is given by:
\begin{equation}
    \omega_i(\tau; \tau_{stall}^{(i)}, \omega_{free}^{(i)}) = \frac{(\tau_{stall}^{(i)} - \tau)\omega_{free}^{(i)}}{\tau_{stall}^{(i)}}
\end{equation}

\begin{equation}
    \tau_1 \geq F(R_1 + R_2 + R_3) + G(M_2 R_1 + M_3 (R_1 + R_2))
\end{equation}

\begin{equation}
    \tau_2 \geq F(R_2 + R_3) + M_3 G R_2
\end{equation}

\begin{equation}
    \tau_3 \geq F R_3
\end{equation}

\begin{equation}
    \omega_1(\tau_1) \geq \frac{V}{R_1 + R_2 + R_3}
\end{equation}

\begin{equation}
    \omega_2(\tau_2) + \omega_3(\tau_3) \geq \frac{V}{R_2 + R_3}
\end{equation}

\begin{equation}
    \omega_3(\tau_3) \geq \frac{V}{R_3}
\end{equation}

\FloatBarrier
\subsection{Optimization Goal}

We want to optimize for price (lowest price).

\end{document}
