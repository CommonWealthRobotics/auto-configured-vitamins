\documentclass{article}
\usepackage[utf8]{inputenc}
\usepackage{hyperref}
\usepackage{amsmath}
\usepackage{bera}
\usepackage{listings}
\usepackage{xcolor}
\usepackage{graphicx}
\usepackage{subcaption}
\usepackage{placeins}
\usepackage[paper=a4paper,margin=1.25in]{geometry}
\usepackage{bm}
\usepackage{amssymb}
\usepackage{interval}

\title{Arm Constraints}
\author{Kevin Harrington, Ryan Benasutti}
\date{\today}

\newcommand{\code}{\texttt}

\begin{document}

\maketitle

\FloatBarrier
\section{Arm Constraints}

\FloatBarrier
\subsection{Features}

The feature matrix of a motor module is given by
\begin{equation}
    F_m =
    \begin{bmatrix}
        \tau_{stall}^{(1)} & \tau_{stall}^{(2)} & \tau_{stall}^{(3)} \\[6pt]
        \omega_{free}^{(1)} & \omega_{free}^{(2)} & \omega_{free}^{(3)} \\[6pt]
        P^{(1)} & P^{(2)} & P^{(3)} \\[6pt]
        M^{(1)} & M^{(2)} & M^{(3)}
    \end{bmatrix}
\end{equation}

where $\tau_{stall}^{(i)}$ is the stall torque in Newton-meters for motor $i$, $\omega_{free}^{(i)}$
is the free speed in radians per second for motor $i$, $P^{(i)}$ is the price of motor $i$, and
$M^{(i)}$ is the mass in kilograms of motor $i$.

\FloatBarrier
\subsection{Required Tip Velocity and Force Throughout Workspace}

$\mathbb{W}^N$ is the workspace of the arm. The tip velocity and tip force
constraints below will be met inside most of the workspace if they are met near
the edges. Configurations on the workspace surface itself are unimportant
because the arm has very little manipulability in these configurations. We
define $\mathbb{W}^{\prime N} \subset \mathbb{W}^N$ as a surface offset from
the edge of the workspace.

We wish to enforce a minimum reachable tip velocity ($V_{tip}$, given as
\texttt{requiredTipForceNewtons}) and tip force ($F_{tip}$, given as
\texttt{requiredTipForceNewtons}) throughout all configurations $\boldsymbol{\theta} \in
\mathbb{W}^N$.

\begin{equation}
    \forall \boldsymbol{\theta} \in \mathbb{W}^{\prime N}, J(\boldsymbol{\theta}) \boldsymbol{\dot{\theta}} \geq V_{tip}
    \label{eq:vtip_constraint}
\end{equation}

\begin{equation}
    \forall \boldsymbol{\theta} \in \mathbb{W}^{\prime N}, J(\boldsymbol{\theta}) \boldsymbol{\tau} \geq F_{tip}
    \label{eq:ftip_constraint}
\end{equation}

$\boldsymbol{\dot{{\theta}}}$ and $\boldsymbol{\tau}$ are linked by the torque-speed curve of a motor,
\begin{equation}
    \tau_j = t^{(i)}(\dot{\theta}_j)
\end{equation}

The torque-speed curve of a motor is given by:
\begin{equation}
    t^{(i)}(\omega) = \tau_{stall}^{(i)} - \frac{\omega \cdot \tau_{stall}^{(i)}}{\omega_{free}^{(i)}}
\end{equation}

When specifying $\boldsymbol{\dot{\theta}}$, any value for $\dot{\theta_j}$
which satisfies both \ref{eq:vtip_constraint} and \ref{eq:ftip_constraint} and
is in the range $\interval{-\omega_{free}^{(i)}}{\omega_{free}^{(i)}}$ is
valid.

\FloatBarrier
\subsection{Optimization Goal}

We want to optimize for price.

\end{document}
